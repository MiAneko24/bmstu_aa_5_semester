\documentclass[../report.tex]{subfiles}

\begin{document}

В процессе выполнения лабораторной работы было реализовано асинхронное взаимодействие потоков на примере конвейерной обработки данных. В качестве конвейера рассмотрена работа с логинами и паролями пользователей: загрузка из базы данных, медленное хеширование (с использование криптографического хеша SHA-512) и сохранение полученных данных в базу данных.

В результате исследования времени выполнения алгоритма было выявлено, что в среднем для количества пользователей от 5 до 85 выигрыш при использования конвейерной обработки составляет 1.6 раз.

\end{document}