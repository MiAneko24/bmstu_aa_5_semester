\documentclass[../report.tex]{subfiles}

\begin{document}
Конвейеризация (конвейерная обработка) -- способ организации вычислений, основанный на разделении подлежащей исполнению функции на более мелкие задачи, называемые ступенями (этапами) конвеера, и выделении для каждой из них отдельного блока. Производительность при этом возрастает благодаря тому, что одновременно на различных ступенях конвейера выполняются несколько задач.

В данной работе рассматривается конвейеризация алгоритма медленного хеширования извлеченных из исходной таблицы базы данных паролей с последующим их сохранением в итоговую таблицу базы.

Целью данной работы является изучение и реализация методов асинхронного взаимодействия потоков для конвейерной обработки данных на материале указанного выше алгоритма.

Для достижения поставленной цели необходимо выполнить следующие задачи:
\begin{enumerate}
	\item Разработать схему конвейерной обработки для указанного алгоритма.
	\item Получить практические навыки реализации последовательной и конвеерной реализаций.
	\item Провести сравнительный анализ последовательной и конвейерной реализации по затрачиваемым ресурсам (времени работы).
\end{enumerate}

\end{document}
