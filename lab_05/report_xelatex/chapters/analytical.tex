\documentclass[../report.tex]{subfiles}

\begin{document}
В данном разделе рассматриваются решаемая задача и выбранный способ её конвейеризации.

\section{Описание решаемой задачи}
Главная часть решаемой задачи состоит в медленном шифровании паролей пользователей. Но перед этим пароли необходимо извлечь из базы данных, а после шифрования сохранить их в эту базу. То есть задача естестественно разбита на 3 этапа. Эти этапы и будут выделены в стадии конвейера.

\section{Выделенные стадии конвейерной обработки}
Было выделено три стадии конвейера:
\begin{enumerate}
	\item Загрузка логина и пароля пользователя из базы данных.
	\item Медленное шифрование пароля.
	\item Вставка логина и зашифрованного пароля в базу данных.
\end{enumerate}

\subsection{Загрузка логина и пароля из базы данных}
На данном этапе логин и незашифрованный пароль пользователя загружаются из базы данных.

\subsection{Медленное шифрование пароля}
Медленное шифрование (хеширование) -- способ шифрования паролей, основанный на последовательном нахождении хеша от значения, полученного на предыдущей стадии шифрования. Чем больше таких стадий (итераций), тем дольше происходит шифрование. Такой способ усложняет проведение брутфорс атаки (перебор значений) для нахождения пароля выбранного пользователя ввиду низкой скорости проверки каждого пароля.

На данном этапе осуществляется медленное хеширование извлеченного из базы пароля.

\subsection{Загрузка в базу данных}
На данном этапе исходный логин и зашифрованный пароль сохраняются в базу данных.

\section{Требования к программному обеспечению}
К рассмотренному алгоритму выдвигаются следующие требования:
\begin{itemize}
	\item Входные данные:
		\begin{enumerate}
			\item База данных с исходной и итоговой таблицами. Таблицы состоят из двух колонок строкового типа (login, password).
		\end{enumerate}
	
	\item Выходные данные:
		\begin{enumerate}
			\item Время работы при последовательном выполнения действий (вещественное число).
			\item Время работы конвейера (вещественное число).
			\item Минимальное, среднее и максимальное времена ожидания заявки в очереди каждой из конвейерных лент (вещественные числа).
		\end{enumerate}
\end{itemize}

\section*{Вывод}
Был рассмотрена решаемая задача. Были определены и описаны этапы конвейерной реализации алгоритма. Были выдвинуты требования к разрабатываемому ПО.
\end{document}
