\documentclass[../report.tex]{subfiles}

\begin{document}
Данный раздел содержит обоснование выбора языка программирования и среды разработки, а также программную реализацию сконструированных алгоритмов.

\section{Средства реализации}
Для реализации программы был выбран язык программирования Python~\cite{python}. Такой выбор обусловлен следующими причинами:
\begin{itemize}
	\item имеются стандартные библиотеки для работы с потоками, хеширования и измерения времени,
	\item высокая скорость разработки.
\end{itemize}

В качестве функции для шифрования будет использоваться криптографический хеш SHA-512. На момент написания работы это хеш без коллизий.

\section{Реализация алгоритмов}
В листингах \ref{lst:user} - \ref{lst:master} представлены реализации рассматриваемых алгоритмов.
\newpage

\begin{lstlisting}[caption=Класс User, label={lst:user}]
	PC = 'qwertyuiopasdfghjklzxcvbnm1234567890'
	PCS = len(PC)
	
	class User:
	def __init__(self, login: str = None, password: str = None):
	if login is None:
	self.__login = str(uuid.uuid4())
	else:
	self.__login = login
	if password is None:
	self.__password = self.__generate_pass()
	else:
	self.__password = password
	
	@staticmethod
	def __generate_pass():
	size = random.randint(16, 2**9)
	a = ''
	for i in range(size):
	a += PC[random.randint(0, PCS - 1)]
	return a
	
	@property
	def login(self):
	return self.__login
	
	@property
	def password(self):
	return self.__password
\end{lstlisting}

\newpage
\begin{lstlisting}[caption=Класс Time]
	class Time:
	def __init__(self):
	self.__time_in = None
	self.__time_out = None
	
	def set(self, t, is_in: bool):
	if is_in:
	if self.__time_in is not None:
	raise Exception("InTime already recorded!")
	self.__time_in = t
	elif not is_in:
	if self.__time_out is not None:
	raise Exception("OutTime already recorded!")
	self.__time_out = t
	else:
	raise Exception("is_in needed!")
	
	def get(self, is_in: bool = None):
	if is_in is None:
	if self.__time_out is None and self.__time_in is None:
	raise Exception("Time's not set!")
	elif self.__time_out is None:
	raise Exception("Can't get out_time!")
	elif self.__time_out is None:
	raise Exception("Can't get out_time!")
	return [self.__time_in, self.__time_out]
	elif is_in:
	if self.__time_out is None:
	raise Exception("Can't get in_time!")
	return self.__time_in
	else:
	if self.__time_out is None:
	raise Exception("Can't get out_time!")
	return self.__time_out
\end{lstlisting}

\newpage
\begin{lstlisting}[caption=Класс UserStats]
	class UserStats:
	cnt = 0
	
	def __init__(self, login: str = None, password: str = None, 	do_not_init: bool = True):
	self.__times = [Time() for _ in range(3)]
	self.__number = self.cnt
	if do_not_init:
	self.user = None
	else:
	self.user = User(login, password)
	self.current = ''
	UserStats.cnt += 1
	
	def set_time(self, time_, stage: int, is_in: bool):
	# is_in - True, in, False, out
	# stage - номер ленты
	if stage < 0 or stage > 2 or stage is None:
	raise Exception("Wrong num!")
	self.__times[stage].set(time_, is_in)
	
	def get_time(self, is_in: bool = None, stage: int = None):
	# is_in - True, False, out
	# stage - num, None = all
	
	if stage is not None and (stage < 0 or stage > 2):
	raise Exception("Wrong num!")
	if stage is None:
	if is_in is None:
	return [self.__times[i].get() for i in range(3)]
	else:
	return [self.__times[i].get(is_in) for i in range(3)]
	else:
	return self.__times[stage].get(is_in)
	
	def get_number(self):
	return self.__number
\end{lstlisting}

\newpage
\begin{lstlisting}[caption=Файл master.py (часть 1)]
	def init_db(con: sqlite3.Connection):
	cur = con.cursor()
	cur.execute('create table if not exists users_out (login text, password text)')
	cur.execute('create table if not exists users_in (i int, login text, password text)')
	con.commit()
	
	def fill_input_db(con: sqlite3.Connection, count=30):
	cur = con.cursor()
	for index in range(count):
	m = User()
	cur.execute(f'insert into users_in (i, login, password) values ({index}, "{m.login()}", "{m.password()}")')
	con.commit()
	
	def get_list_size_from_db(con: sqlite3.Connection) -> int:
	cur = con.cursor()
	cur.execute('select count(*) from users_in')
	return cur.fetchone()[0]
	
	def clear_db(con: sqlite3.Connection):
	cur = con.cursor()
	cur.execute('drop table if exists users_out')
	cur.execute('drop table if exists users_in')
	con.commit()
	
	def generate_users(users_count: int) -> [UserStats]:
	users = []
	
	for i in range(users_count):
	users.append(UserStats(do_not_init=False))
	
	return users
\end{lstlisting}

\newpage
\begin{lstlisting}[caption=Файл master.py (часть 2)]
	def load_user(u: UserStats):
	con = sqlite3.connect('app.db')
	c = con.cursor()
	c.execute(f'select login, password from users_in where i = {u.get_number()}')
	login, password = c.fetchone()
	u.user = User(login, password)
	
	def get_hashed(u: UserStats):
	u.current = u.user.login() + 'my secret key'
	
	for iter in range(2000):
	u.current = sha512((u.current + str(iter)).encode('Utf-8')).hexdigest()
	
	def insert(u: UserStats):
	con = sqlite3.connect('app.db')
	c = con.cursor()
	c.execute(f"insert into users_out (login, password) values ('{u.user.login()}', '{u.current}')")
	con.commit()
	con.close()
	
	def tex_table(stats, stages):
	print("\\hline")
	print('Stage N & Task M & Start Time & End Time\\\\')
	print("\\hline")
	
	for stat_num, stat in enumerate(stats):
	for stage in range(stages):
	times = stat.get_time(stage=stage)
	print(
	f'Stage: {stage + 1} & Task: {stat_num + 1} & {times[0] - start_time:.6f} & {times[1] - start_time:.6f} \\\\')
	
	print("\\hline")
\end{lstlisting}

\newpage
\begin{lstlisting}[caption=Файл master.py (часть 3)]
	def job(task: Callable, in_queue: SimpleQueue, out_queue: SimpleQueue, stage: int):
	while True:
	data: UserStats = in_queue.get()
	
	if data is None:
	out_queue.put(data)
	break
	
	data.set_time(time.time(), stage, True)
	task(data)
	data.set_time(time.time(), stage, False)
	out_queue.put(data)
	
	def test_serial(users):
	for user in users:
	load_user(user)
	get_hashed(user)
	insert(user)
	
	if __name__ == '__main__':
	stages_count = 3
	
	connection = sqlite3.connect('app.db')
	clear_db(connection)
	init_db(connection)
	fill_input_db(connection)
	
	_users = generate_users(get_list_size_from_db(connection))
	
	pipeline_time = 0
	serial_time = 0
	cnt = 10
	for i in range(cnt):
	clear_db(connection)
	init_db(connection)
	fill_input_db(connection)
	
	passwords_queue = SimpleQueue()
	salt_queue = SimpleQueue()
\end{lstlisting}

\newpage
\begin{lstlisting}[caption=Файл master.py (часть 4)]
	hash_queue = SimpleQueue()
	result_queue = SimpleQueue()
	
	add_salter = Process(target=job, args=(load_user, passwords_queue, salt_queue, 0))
	hasher = Process(target=job, args=(get_hashed, salt_queue, hash_queue, 1))
	inserter = Process(target=job, args=(insert, hash_queue, result_queue, 2))
	pipeline = [add_salter, hasher, inserter]
	
	for u in _users:
	passwords_queue.put(u)
	
	passwords_queue.put(None)
	start_time = time.time()
	for worker in pipeline:
	worker.start()
	
	for worker in pipeline:
	worker.join()
	end_time = time.time()
	pipeline_time += end_time - start_time
	
	start_time_ = time.time()
	test_serial(_users)
	end_time_ = time.time()
	serial_time += end_time_ - start_time_
	
	pipeline_time /= cnt
	
	print(f'Pipeline time = {pipeline_time * 1e6} mks')
	
	serial_time /= cnt
	print(f'Serial time = {serial_time * 1e6} mks')
	print('Press to get log')
	input()
	
	stats = []
\end{lstlisting}
\newpage
\begin{lstlisting}[caption=Файл master.py (часть 5), label={lst:master}]
	while not result_queue.empty():
	stats.append(result_queue.get())
	stats.pop()
	
	deltas = [[], [], []]
	for stat in stats:
	for stage in range(stages_count):
	stage_stat = stat.get_time(stage=stage)
	deltas[stage].append(stage_stat[1] - stage_stat[0])
	
	for stage in range(stages_count):
	print(f'Max time on stage {stage + 1} = {max(deltas[stage]) * 1e6} mks')
	print(f'Min time on stage {stage + 1} = {min(deltas[stage]) * 1e6} mks')
	print(f'Avg time on stage {stage + 1} = {sum(deltas[stage]) / len(deltas[stage]) * 1e6} mks\n')
	
	connection.close()
	
	tex_table(stats, stages_count)
\end{lstlisting}


\section*{Вывод}
В данном разделе была реализована конвейерная обработка данных: извлечение из базы данных, хеширование и сохранение паролей пользователей.
\end{document}
